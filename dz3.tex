\documentclass[a4paper, 12pt]{article}
\usepackage[utf8]{inputenc}
\usepackage[english, russian]{babel} 
\usepackage[left = 3cm, right = 1cm, top = 2cm, bottom = 2cm]{geometry};
\usepackage{amsfonts}
\usepackage[usenames]{color}
\usepackage{mathtools, graphicx}
\usepackage{amssymb}
\usepackage{listings}

\linespread{1.25}

\title{ДЗ № 3: Машины Тьюринга и квантовые вычисления.}
\author{Сингин Александр. Группа А-13б-19.}
\date{}

\begin{document}

\maketitle

\newpage

\section{Введение}

\section{Машины Тьюринга}

\subsection{Операции с числами}

\begin{enumerate}

\itemСложение двух унарных чисел

\begin{verbatim}
name: binary increment
source code: |+
  input: '111+11'
  blank: ' '
  start state: find_plus
  table:
    # Идем вправо до "+"
    find_plus:
      1: R
      +: {write: 1, R: write_plus}
      
    # Заменяем "+" на "1"
    write_plus:
      1: {write: +, L: find_plus}
      ' ': {L: last}
      
    # Удаляем посл. "1"
    last:
      1: {write: ' ', L: done}
      
    done:
positions:
  find_plus: {x: 275.17, y: 388.16, fixed: false}
  write_plus: {x: 366.02, y: 286.98}
  last: {x: 508.2, y: 247.79, fixed: false}
  done: {x: 672, y: 261.09}
\end{verbatim}

\textcolor{green}{Пример:}

\newpage
Введено: 111+11

\begin{figure}[!h]
\centering
\includegraphics[scale = 0.3]{2.1_1.jpg}
\end{figure}

Получено:

\begin{figure}[!h]
\centering
\includegraphics[scale = 0.3]{2.1_1a.jpg}
\end{figure}

\itemУмножение унарных чисел

\begin{verbatim}
name: binary increment
source code: |-
  input: '111*11='
  blank: ' '
  start state: start_mul
  table:
    start_mul:
      1: {write: x, R: find_second}
      x: R
      '*': {L: left_end}
      
    # Идем ко 2-ому слагаемому
    find_second:
      1: R
      '*': {R: x_second}
      
    # Обработка 2-ого слагаемого
    x_second:
      x: R
      1: {write: x, R: write_in_answer}
      =: {L: recover_second}
      
    # Приписываем "1" в ответ
    write_in_answer:
      [1, =]: R
      ' ': {write: 1, L: back_to_second}
      
    # Возвращаемся ко 2-ому слагаемому
    back_to_second:
      [1, =]: L
      x: {R: x_second}
      
    # Восстанавливаем 2-ое слагаемое (Н-р, xx -> 11)
    recover_second:
      x: {write: 1, L}
      '*': {L: back_to_first}
      
    # Возвращаемся к 1-ому слагаемому; обр. остав. разряды 1-ого слагаемого
    back_to_first:
      1: L
      x: {R: start_mul}
      
    # Возвращаемся в начало
    left_end:
      x: L
      ' ': {R: del}
      
    # Удаляем входные данные
    del:
      [1, '*', x]: {write: ' ', R: del}
      =: {write: ' ', R: done}
      
    done:
positions:
  start_mul: {x: 375.1, y: 89.77, fixed: false}
  find_second: {x: 334.98, y: 249.53, fixed: false}
  x_second: {x: 444.97, y: 339.81, fixed: false}
  write_in_answer: {x: 503.66, y: 469.97, fixed: false}
  back_to_second: {x: 360.58, y: 456.42, fixed: false}
  recover_second: {x: 376.72, y: 203.31, fixed: false}
  back_to_first: {x: 239.46, y: 122.74, fixed: false}
  left_end: {x: 288.64, y: 197.28, fixed: false}
  del: {x: 448.86, y: 162.93, fixed: false}
  done: {x: 570, y: 250}
\end{verbatim}

\textcolor{green}{Пример:}

Введено: 111*11=

\begin{figure}[!h]
\centering
\includegraphics[scale = 0.3]{2.1_2.jpg}
\end{figure}

\newpage
Получено:

\begin{figure}[!h]
\centering
\includegraphics[scale = 0.3]{2.1_2a.jpg}
\end{figure}

\end{enumerate}

\subsection{Операции с языками и символами}

Реализуйте машины Тьюринга, которые позволяют выполнять следующие операции:

\begin{enumerate}

\itemПринадлежность к языку $L = \{0^n1^n2^n\}, n \geq 0$

\begin{verbatim}
name: binary increment
source code: |
  input: '001122'
  blank: ' '
  start state: start
  table:
    start:
      1: {R: checkNumOne}
      0: {write: a, R: checkOne}
      ['a']: R
      ['b']: R
      ['c']: R
      ' ': {L: check}
    checkNumOne:
      1: R
      ' ': {write: '!', R: done}
    checkOne:
      1: {write: b, R: checkTwo}
      0: R
      ['b']: R
    checkTwo: 
      2: {write: c, L: backStart}
      1: R
      ['c']: R
    backStart:
      [0, 1, 2, 'a', 'b', 'c']: L
      ' ': {R: start}
    check:
      ['a', 'b', 'c']: L
      ' ': {write: '!', L: return}
    return:
      [0, 1, 2, 'a', 'b', 'c']: L
      ' ': {R: done}
    done:
positions:
  start: {x: 432.84, y: 280.39}
  checkNumOne: {x: 499.75, y: 387.63}
  checkOne: {x: 293.72, y: 169.31}
  checkTwo: {x: 169.91, y: 311.54}
  backStart: {x: 353.72, y: 438.73}
  check: {x: 401.42, y: 65.42}
  return: {x: 548.8, y: 29.33}
  done: {x: 681.85, y: 330.26}

\end{verbatim}
\includegraphics[scale = 0.3]{2.2_1.jpg}


\newpage

\itemПроверка соблюдения правильности скобок в строке (Минимум 3 вида скобок).

\begin{verbatim}
input: '()[]'
blank: ' '
start state: start
table:
  start:
    ' ': {L: ok}
    ['(','[','{']: {R: find-closed}
    [')',']','}']: {L: not-ok}
  
  find-closed:
    ' ': {L: empty-or-ok}
    ['(','[','{','x']: R
    ')': {write: 'x', L: closed_1}
    ']': {write: 'x', L: closed_2}
    '}': {write: 'x', L: closed_3}
  
  closed_1:
    ' ': {L: not-ok}
    '(': {write: 'x', R: find-closed} 
    ['[','{']: {L: not-ok}
    'x': L
  
  closed_2:
    ' ': {L: not-ok}
    '[': {write: 'x', R: find-closed} 
    ['(','{']: {L: not-ok}
    'x': L
  
  closed_3:
    ' ': {L: not-ok}
    '{': {write: 'x', R: find-closed} 
    ['[','(']: {L: not-ok}
    'x': L
  
  empty-or-ok:
    ['(','[','{']: {L: not-ok}
    'x': L
    ' ': {R: ok}
  
  not-ok:
    ['(',')','[',']','{','}','x']: {write: ' ', R}
    ' ': {R: go-start}
  
  go-start:
    ['(',')','[',']','{','}','x']: {write: ' ', R: go-start}
    ' ': {write: 0, L: done}
  
  ok:
    ' ': {write: 1, L: done}
    'x': {write: ' ', R}
  
  done:

\end{verbatim}



\begin{figure}[!h]
\centering
\includegraphics[scale = 0.3]{2.2_2.jpg}
\end{figure}

\newpage

\itemПоиск минимального по длине слова в строке (Слова состоят из символов 1 и 0 и разделены пробелом).

\begin{verbatim}
name: 'binary increment'
source code: |
  input: '10101 101 100'
  # input: '11 01 10'
  # input: '1 101 110'
  # input: '1'
  blank: ' '
  start state: q0
  table:
    q0:
      [1, 0]: {L}
      ' ': {write: '#', R: q1}
    q1:
      [1, 0]: {R}
      ' ': {R: q2}
    q2:
      [1, 0]: {R: q1}
      ' ': {write: '*', L: q3}
    q3:
      [1, 0, '|', 'O', ' ']: {L}
      '#': {R: replace}
    replace:
      ['|', 'O']: {R}
      1: {write: '|', R: next}
      0: {write: 'O', R: next}
      ' ': {L: overwrite}
      '*': {L: q3}
    next:
      [1, 0]: {R}
      ' ': {R: replace}
      '*': {L: q3}
    overwrite:
      '|': {write: 'B', L}
      'O': {write: 'A', L}
      ' ': {write: '&', L: delL}
      '#': {write: '&', R: delR}
    delL:
      ' ': {L}
      [1, 0, '|', 'O']: {write: ' ', L}
      '#': {write: ' ', R: delR}
    delR:
      [' ', 'A', 'B', '&']: {R}
      [1, 0, '|', 'O']: {write: ' ', R}
      '*': {write: ' ', L: goto}
    goto:
      ' ': {L}
      'A': {write: 0, L}
      'B': {write: 1, L}
      '&': {write: ' ', R: done}
    done:
positions:
  q0: {x: 81.08, y: 73.12}
  q1: {x: 251.31, y: 72.3}
  q2: {x: 470.63, y: 79.05}
  q3: {x: 473.86, y: 207.64}
  replace: {x: 265.19, y: 320.03}
  next: {x: 71.7, y: 205.53}
  overwrite: {x: 450.05, y: 380.05}
  delL: {x: 61.09, y: 475.98}
  delR: {x: 684.06, y: 460.28}
  goto: {x: 780, y: 460.8}
  done: {x: 780, y: 44.75}
editor contents: |
  input: '10101 101 100'
  blank: ' '
  start state: q0
  table:
    q0:
      [1, 0]: {L}
      ' ': {write: '#', R: q1}
    q1:
      [1, 0]: {R}
      ' ': {R: q2}
    q2:
      [1, 0]: {R: q1}
      ' ': {write: '*', L: q3}
    q3:
      [1, 0, '|', 'O', ' ']: {L}
      '#': {R: replace}
    replace:
      ['|', 'O']: {R}
      1: {write: '|', R: next}
      0: {write: 'O', R: next}
      ' ': {L: overwrite}
      '*': {L: q3}
    next:
      [1, 0]: {R}
      ' ': {R: replace}
      '*': {L: q3}
    overwrite:
      '|': {write: 'B', L}
      'O': {write: 'A', L}
      ' ': {write: '&', L: delL}
      '#': {write: '&', R: delR}
    delL:
      ' ': {L}
      [1, 0, '|', 'O']: {write: ' ', L}
      '#': {write: ' ', R: delR}
    delR:
      [' ', 'A', 'B', '&']: {R}
      [1, 0, '|', 'O']: {write: ' ', R}
      '*': {write: ' ', L: goto}
    goto:
      ' ': {L}
      'A': {write: 0, L}
      'B': {write: 1, L}
      '&': {write: ' ', R: done}
    done:

\end{verbatim}

\begin{figure}[!h]
\centering
\includegraphics[scale = 0.3]{2.2_3.jpg}
\end{figure}
\end{enumerate}
\newpage

\section{Квантовые вычисления}


Для выполнения заданий по квантовым вычислениям требуется QDK. Его можно скачать здесь: \url{https://docs.microsoft.com/en-us/azure/quantum/install-overview-qdk}. 
\\\\
Но можно использовать любой пакет, типа \url{https://qiskit.org/}. 
\\\\
В качестве решения задачи надо предоставить схему алгоритма для частного случая при фиксированном количестве кубитов и фиксированных состояниях. 


\subsection{Генерация суперпозиций 1 (1 балл)}

Дано $N$ кубитов ($1 \le N \le 8$) в нулевом состоянии $\Ket{0\dots0}$. Также дана некоторая последовательность битов, которое задаёт ненулевое базисное состояние размера $N$. Задача получить суперпозицию нулевого состояния и заданного.

$$\Ket{S} = \frac{1}{\sqrt2}(\Ket{0\dots0} +\Ket{\psi})$$

То есть требуется реализовать операцию, которая принимает на вход:

\begin{enumerate}
    \item Массив кубитов $q_s$
    \item Массив битов $bits$ описывающих некоторое состояние $\Ket{\psi}$. Это массив имеет тот же самый размер, что и $qs$. Первый элемент этого массива равен $1$.
\end{enumerate}
\\\\
Заготовка для кода:
\begin{lstlisting}
namespace Solution {
        open Microsoft.Quantum.Primitive;
        open Microsoft.Quantum.Canon;
        operation Solve (qs : Qubit[]) : Int
        {
            body
            {

                return 
            }
        }
}
\end{lstlisting}

\noindentПрименяем к 1-ому кубиту оператор Адамара. Остальные к остальным кубитам, если они равны $ \Ket{1} $ применяем $ CX $, спутывая с первым.

\begin{verbatim}
circuit.h(0)
circuit.barrier()
for i in range(1, len(bits)):
    if bits[i]: circuit.cx(qr[0], qr[i])

circuit.draw(initial_state=True)
\end{verbatim}

\subsection{Различение состояний 1 (1 балл)}

Дано $N$ кубитов ($1 \le N \le 8$), которые могут быть в одном из двух состояний:

$$\Ket{GHZ} = \frac{1}{\sqrt2}(\Ket{0\dots0} +\Ket{1\dots1})$$
$$\Ket{W} = \frac{1}{\sqrt N}(\Ket{10\dots00}+\Ket{01\dots00} + \dots +\Ket{00\dots01})$$

Требуется выполнить необходимые преобразования, чтобы точно различить эти два состояния. Возвращать $0$, если первое состояние и 1, если второе. 
\\\\
Заготовка для кода:
\begin{lstlisting}
namespace Solution {
        open Microsoft.Quantum.Primitive;
        open Microsoft.Quantum.Canon;
        operation Solve (x : Qubit[], y : Qubit, b : Int[]) : ()
        {
            body
            {

            }
        }
}
\end{lstlisting}

\noindent Для различения состояний достаточно измерить кубиты. Тогда, если было состояний $GHZ$, все кубиты будут в состоянии $0$ или $1$. Если же было состояние $W$, то только один кубит был в состоянии $1$

В случае, когда $N = 1$, состояние не различить, так как в обоих случая может быть состояние $|1\rangle$

\begin{verbatim}
N = int(input('N = '))
ghz = [1/math.sqrt(2)]
for i in range (1, 2**N-1):
    ghz.append(0)
ghz.append(1/math.sqrt(2))
ghz = Statevector(ghz) 
ghz.draw('latex')
w = [0]
for i in range (1, 2**N):
    if i & i-1:
        w.append(0)
    else:
        w.append(1/math.sqrt(N))
w = Statevector(w)
w.draw('latex')
def Solve(state):
    res = state.measure()[0]
    c = res.count('1')
    return 1 if c == 1 else 0
res = ''
if Solve(w) == 1: 
    res = "W"
else: res = 'GHZ'
print(f'Ответ: {res}')
\end{verbatim}

\end{document}